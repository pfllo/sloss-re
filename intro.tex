\section{Introduction}

What is relation extracion (RE)?
Why is RE important (e.g., basis of knowledge base population, deeper understand of a text)?
How is RE often appraoched? Give an example.


Apart from the textual information in the sentence, prior knowledge on the constraints of each relation is also important.
Follow the previous example, explain why constraints is useful.
However, most of the existing RE methods ignores this information.
Probably because of no off-the-shelf resources for these clues.

How these constraints are handled by clw?
What's the weakness of clw~\cite{chen2014encoding}? 
Weakness: 
(1) Need postprocessing on a large number of predictions $\rightarrow$ high delay.
(2) Time consuming for large numer of predictions.
(3) clw's model operates on sentence level, which kind of deviates the multi-instance learning procedure, and therefore may introduce some performance drop.

We propose to incorporate these constraints by introducing an additional loss during training, and the test phase incurs no extra costs.
Specifically, we use the semantic loss framework to convert the constraints into a loss term.
The loss penalizes inconsistent predictions during training.
In this way, the classification boundary is made more clear and discriminative $\rightarrow$ better generalization.
Since we only add a loss term, our framework is compatible to most of the relation extractors that uses gradient descent.
Further, the extra cost only reside in the training procedure, and the test is as efficient as before.
We conduct experiments on English and Chinese datasets.
The experimental results show that our method can clearly improve the base model, and is superior to clw's method.
Mention the simplified version?

